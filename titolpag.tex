\begin{titlepage}
\sffamily
\setlength{\unitlength}{\textwidth}
\begin{picture}(1,1.41)              % pagxforma
\thinlines
\put(0.05,0.05){\line(0,1){1.31}}         % vertikala linio maldekstra
\put(0.95,0.05){\line(0,1){1.31}}         % vertikala linio dekstra
\put(0.05,0.05){\line(1,0){0.9}}            % horizontala linio malsupra
\put(0.05,1.36){\line(1,0){0.9}}         % horizontala linio supra
\thicklines
\put(0,0){\line(0,1){1.41}}         % vertikala linio maldekstra
\put(1,0){\line(0,1){1.41}}         % vertikala linio dekstra
\put(0,0){\line(1,0){1}}            % horizontala linio malsupra
\put(0,1.41){\line(1,0){1}}         % horizontala linio supra
\put(0.5,1.2){   \makebox(0,0){\huge  Nescio}}
\put(0.3,1.1){\line(1,0){0.4}}
\put(0.5,1.0){ \makebox(0,0){\Huge la profitulo}    }
\end{picture}
\end{titlepage}
\rmfamily
\pagestyle{empty}
%\vspace*{\textheight}
\hbox{}
\vfill
\begin{minipage}[t]{\textwidth}
%\today\\

Traduko de la nederlandlingva rakonto {\em De uitvreter}, kiu unuafoje
estis publikigita en la revuo {\em De Gids} en januaro de la jaro
1911. La a\u{u}toro estas J.H.F. Gr\"onloh, pse\u{u}donomita Nescio, kiu
vivis de 1882 \^gis 1961.  La traduko estis farita ek de 1999 far Michiel
Meeuwissen $<$mihxil@gmail.com$>$, kun iom helpo de aliaj.
% Tiu \^ci versio tamen ankora\u{u} estu rigardata kiel versio por provlegantoj.
% Anka\u{u} tial la alineoj estas numeritaj. Tio espereble faciligos komenti.
\\
La plej nova versio de tiu \^ci traduko \^ciam trovi\^gas je:\\
http://www.purl.org/NET/mihxil/profitulo/\\
\\
Enpa\^gigita per \LaTeX
\end{minipage}
\newpage
\pagestyle{plain}
\setcounter{page}{1}
